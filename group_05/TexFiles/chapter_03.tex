\chapter{Notes for the resume}
\setlength{\headheight}{12.71342pt}
\addtolength{\topmargin}{-0.71342pt}


\section{Introduction}
\begin{itemize}
    \item Phospholipids (PL) in Milk: Estimates of PL content in bovine milk vary (0.9–2.3\% of total lipids) depending on extraction methods (Röse Gottlieb vs. Folch extraction).
    \item Milk Fat Globule Membrane (MFGM): Contains 60–65\% of milk PL, with the remaining 35–40\% found in the skim milk phase.
    \item Extracellular Vesicles (EVs): Recently recognized nano-sized phospholipid structures in skim milk, distinct from milk fat globules, but involved in intercellular communication.
    \item Membrane-Associated Enzymes: Identified in both MFGs and EVs, but their abundance is generally low. Many originate from ER, Golgi, or cytosolic crescents.
    \item Activity Considerations: Many of these enzymes remain inactive in milk due to the absence of substrates or unsuitable environmental conditions.
    \item Scope of Discussion: The summary will focus on enzymes relevant to mammary gland biology, milk integrity, and physiological functions upon consumption, excluding those related to lipid synthesis.
\end{itemize}

\section{Sulfhydryl Oxidase}
\begin{itemize}
    \item Sulfhydryl Oxidase (EC 1.8.3.2): Catalyzes oxidation of protein thiols (cysteine residues) to form disulfide bonds, reducing oxygen to hydrogen peroxide.
    \item Types in Milk: Exists in metal-dependent and flavin-dependent forms.
    \item Early Studies:
    \subitem Iron-dependent sulfhydryl oxidase (89-kDa, contains iron) was initially reported (Janolino \& Swaisgood, 1975).
    \subitem Later studies failed to confirm its presence (Jaje et al., 2007).
    \item Current Understanding:
    \subitem Flavin-dependent sulfhydryl oxidase (QSOX1) is well-documented in bovine milk.
    \subitem Sequence analysis confirmed it as part of the Quiescin-sulfhydryl oxidase family.
    \item Membrane Association:
    \subitem Initially believed to be strongly associated with phospholipid membranes (Kitchen, 1974).
    \subitem Later studies suggest a looser association, making it a more soluble protein (Jaje et al., 2007).
    \item Proteomic Evidence: 
    \subitem QSOX1 has been identified in membrane fractions of both human and bovine milk
\end{itemize}

\subsection*{Structure of Flavin-Dependent Sulfhydryl Oxidase}
\begin{itemize}
    \item QSOX1 Splice Variants:
    \subitem QSOX1-L (Long form, 79.6 kDa): Contains a transmembrane region.
    \subitem QSOX1-S (Short form, 63.8 kDa): Lacks most of exon 12 and is more soluble.
    \subitem QSOX1-S is more prevalent than QSOX1-L, including in mammary-derived cell lines.

    \item Structural Features:
    \subitem Multi-domain enzyme derived from fusion of two ancient genes.
    \subitem Contains thioredoxin (Trx) domains, FAD-binding module, CxxC motifs (common in redox reactions).
    \subitem QSOX1-L has a membrane-spanning region, while QSOX1-S does not.

    \item QSOX1 in Bovine Milk:
    \subitem Jaje et al. (2007) isolated sulfhydryl oxidase from bovine skim milk, identifying it as QSOX1-L, though later studies suggest it was likely QSOX1-S.
    \subitem The enzyme migrated as a 62 kDa band in SDS-PAGE.

    \item Biological Significance:
    \subitem QSOX1 expression is linked to tumorigenesis (Antwi et al. 2009; Katchman et al. 2013).
    \subitem QSOX1-S has been isolated from mammalian blood serum (Israel et al. 2014).
    \subitem QSOX1-L can be proteolytically modified and secreted into the extracellular matrix, possibly after removal of its transmembrane domain (Rudolf et al. 2013).
    \subitem Dimerization of QSOX1 has been demonstrated in the same study.
\end{itemize}

\subsection*{Biological Role of Flavin-Dependent Sulfhydryl Oxidase and Significance in Milk}

\begin{itemize}
    \item QSOX1 in High Secretory Cells:
    \subitem Associated with Golgi membranes, cell surface, and secreted as an enzyme.
    \subitem Overexpressed in breast tissue, pancreas, and prostate cancer, suggested as a diagnostic cancer marker.
    \subitem Also linked to heart failure diagnosis.
    
    \item Function in Milk:
    \subitem May contribute to protein folding, but exact substrates remain unidentified.
    \subitem "Cooked" and "flat" off-flavou
    rs in UHT milk are associated with sulfur-containing compounds, methyl ketones, and aldehydes.
    \subitem QSOX1 may impact oxidation of these compounds; 60\% of its activity remains post-pasteurization.
    \subitem QSOX immobilized on glass beads has been used to reduce UHT milk off-flavours.

    \item Catalytic Properties:
    \subitem Activity measured using GSH or reduced RNase at pH 7, producing a yellow product (412 nm absorption).
    \subitem Oxidation of thiol substrates generates hydrogen peroxide aerobically; ferricenium ion acts as an alternative electron acceptor anaerobically.
    \subitem Reduction with dithionite or dithiothreitol forms a two-electron intermediate (EH2) with a charge transfer band at 560 nm.
    \subitem Redox-active disulfide bridge involved in catalysis, with similarities to pyridine nucleotide dependent disulfide oxidoreductases.

    \item Comparison with Other QSOX1 Enzymes:
    \subitem Bovine milk QSOX1 has slightly higher kcat/Km values than the chicken egg white enzyme for oxidation of DTT, glutathione, and reduced RNase.
    \subitem Differences may stem from variations in electron flow mechanisms.
\end{itemize}
\cite*{RM_01}

\section{Catalse}
\begin{itemize}
    \item Catalase (EC 1.11.1.6):
    \subitem Catalyzes hydrogen peroxide breakdown into water and oxygen (2H$_2$O$_2$ → 2H$_2$O + O$_2$).
    \subitem Found in most aerobic organisms, primarily located in peroxisomes.

    \item Catalase in Milk:
    \subitem One of the first enzyme activities detected in milk due to easy peroxide monitoring.
    \subitem Historically classified as an indigenous MFGM enzyme, but proteomic data do not confirm its presence in MFGM.
    \subitem Mainly derived from non-milk-secreting cells, present in milk primarily during mastitis.

    \item Reassessment of Catalase in Milk:
    \subitem Likely originates from somatic milk cells or microorganisms (added/contaminating).
    \subitem Not a true membrane-associated enzyme, but included in discussion due to historical classification.
\end{itemize}

\subsection*{Structure of Catalase}
\begin{itemize}
    \item Bovine Liver Catalase:
    \subitem Reliable amino acid sequence (526 amino acids) published by Schroeder et al. (1982).
    \subitem Crystallization studies reported, including a 2.5 Å resolution structure (Murthy et al. 1981).
    \subitem Sequence data available in Uniprot entry P00432.

    \item Catalase in Bovine Milk:
    \subitem Purified from bovine milk by Ito and Akuzawa (1983a, b).
    \subitem Differences exist between bovine and human catalase, but functionally important amino acid regions are conserved.
    \subitem Structurally similar across mammalian species.

    \item Structural Features:
    \subitem Homotetramer of 60-kDa subunits.
    \subitem Belongs to the monofunctional catalase group.
    \subitem Each subunit contains a heme (ferriprotoporphyrin IX) at the active site.
    \subitem Substrate access limited to small molecules.

    \item Secretion and Localization:
    \subitem Encoded as a precursor without a signal sequence.
    \subitem Mature protein has a blocked N-terminal, indicating it is not destined for secretion.
\end{itemize}

\subsection*{Biological Role of Catalase and Significance in Milk}

\begin{itemize}
    \item Catalase as a Mastitis Indicator:
    \subitem Catalase activity has been proposed as a marker for mastitis detection (Kitchen 1976, 1981; Fitz-Gerald et al. 1981).
    \subitem Correlation between catalase activity and bacterial counts can be used in biosensor scans for microbial-challenged milk (Zhang et al. 2014).
    \subitem Traditional confirmatory tests are still required for mastitis diagnosis.

    \item Functional Role of Catalase in Milk:
    \subitem Catalyzes a hydrogen peroxide-dependent reaction converting nitrite to nitrate (Silanikove et al. 2014).
    \subitem Helps maintain low levels of free radicals and oxidation products in milk.
    \subitem Active in the udder, contributing to early milk quality maintenance.
    \subitem Prevents accumulation of oxidation products post-milking and during storage.
    \subitem Inhibition of catalase increases nitrotyrosine and lipid peroxides.

    \item Stability and Heat Sensitivity:
    \subitem Catalase is highly heat-sensitive, similar to alkaline phosphatase (Hirvi et al. 1996).
    \subitem Activity is significantly reduced after heating at 72 \textdegree C for 15 s (Griffiths 1986).
    \subitem Some evidence suggests catalase activity increases after pasteurization due to release from somatic cells and bacteria.
    \subitem Variability in catalase activity may contribute to challenges in raw milk cheese ripening (Gatti et al. 2014; Yoon et al. 2016).
    \subitem Due to post-pasteurization variability, catalase is unsuitable as a pasteurization index.

    \item Catalase Isolation and Structural Properties:
    \subitem First isolated from milk by Ito and Akuzawa (1983a), purified 23,000-fold and crystallized.
    \subitem Three isozymes were identified in milk catalase (Ito and Akuzawa 1983b).
    \subitem SDS dissociation showed subunits ranging from 11 to 55 kDa.
    \subitem Similar in structure to bovine liver catalase, a homotetramer of 60–65 kDa (total mass 250 kDa).
    \subitem Heterogeneity in milk catalase likely results from proteolysis during isolation.

    \item Catalase Activity Measurement:
    \subitem Traditionally measured by monitoring H$_2$O$_2$ disappearance at 240 nm (Beers and Sizer 1952).
    \subitem New and more sensitive assay methods continue to be developed (Hadwan 2018).
    \subitem Commercially available assays exist for precise catalase activity quantification.
\end{itemize}

\section{Lactoperoxidase}
\begin{itemize}
    \item Lactoperoxidase (LPO) (EC 1.11.1.7):
    \subitem A glycoprotein found in exocrine secretions such as saliva, tears, and milk.
    \subitem Belongs to the family of heme-containing peroxidases, also found in plants and fungi.
    \subitem In mammalian peroxidases, the heme group is covalently bound, unlike in fungal and plant peroxidases (Sharma et al. 2013).

    \item Catalytic Function:
    \subitem Catalyzes the breakdown of hydrogen peroxide, generating water (H$_2$O$_2$ + reduced acceptor → oxidized acceptor + 2H$_2$O).
    \subitem The acceptor can be a phenolic compound, aromatic amine, aromatic acid, thiocyanate, bromide, or iodine (Kohler and Jenzer 1989).

    \item Detection and Historical Aspects:
    \subitem LPO activity is easily monitored and was one of the first enzymes described in milk (Arnold 1881).
    \subitem Many chromogenic substrates have been used, but ABTS is currently the most common.
    \subitem Historical aspects of LPO activity in milk have been reviewed (Fox and Kelly 2006).

    \item Presence in Milk Fractions:
    \subitem Traditionally assigned to skim milk and whey.
    \subitem Small amounts detected in MFGM and EVs, likely due to cytosolic contamination.
    \subitem More MS-spectral counts for LPO were found in EVs compared to MFGM in an iTRAQ proteomic study (Reinhardt et al. 2013).

    \item Classification Considerations:
    \subitem LPO is not considered a true membrane-associated milk enzyme.
    \subitem It is still included in discussions as it may be present in phospholipid-containing milk fractions.
\end{itemize}

\subsection*{Structure of Lactoperoxidase}
\begin{itemize}
    \item Bovine Lactoperoxidase (LPO) Sequence:
    \subitem Complete amino acid sequence published in 1990–1991 (Dull et al. 1990; Cals et al. 1991).
    \subitem Mature protein consists of 612 residues after cleavage of signal and propeptides.
    \subitem 3D structure resolved at 2.3 Å (Singh et al. 2009).

    \item Structural and Functional Aspects:
    \subitem Database entry (UniProt P80025) details disulphide bridges, glycosylations, phosphorylations, and heme-binding.
    \subitem Encoded as a secreted protein; no indication of membrane association.

    \item Purification and Analysis:
    \subitem Various methods include salting-out, membrane filtration, chromatography, and immune-affinity techniques.
    \subitem Recent approach using salting-out and ion-exchange yielded high purity (Li et al. 2019).
    \subitem Purity assessed by absorbance at 412 nm (heme content) and A$_{280}$ nm (total protein) (Andersson et al. 1996).
\end{itemize}

\subsection*{Biological Role of Lactoperoxidase and Significance in Milk}
\begin{itemize}
    \item The LPO System:
    \subitem Comprises LPO, hydrogen peroxide, and oxidized products, forming a non-specific humoral immune response (Ihalin et al. 2006).
    \subitem Uses H$_2$O$_2$ to oxidize thiocyanate, iodides, and bromides into antimicrobial hypothiocyanite, hypoiodides, and hypobromides.
    \subitem Inhibits bacterial, fungal, and viral growth by oxidizing thiol groups in proteins (Reiter and Härnulv 1984).

    \item LPO in Milk:
    \subitem Functions in neonatal defence against digestive pathogens (Koksal et al. 2016).
    \subitem Activity in milk debated; requires added thiocyanate and H$_2$O$_2$ for activation (Silanikove et al. 2006).
    \subitem Reacts with nitrite to produce nitric dioxide (NO$_2^-$), involving xanthine oxidase and catalase.
    \subitem Second most abundant enzyme in milk, constituting $\sim$0.5\% of whey proteins (30 mg L$^{-1}$).
    \subitem LPO reported absent in human milk (Hamosh 1988), but later studies confirmed its presence (Shin et al. 2001).

    \item Pasteurization and LPO Testing:
    \subitem LPO loses activity upon heat treatment (Arnold 1881).
    \subitem Basis for the Storch test, used to verify super-pasteurized milk ($\geq$ 76 \textdegree C for 15 s) (Storch 1898).

    \item Industrial and Food Applications:
    \subitem Used in dairy to inhibit microbial growth and preserve raw milk (Seifu et al. 2005).
    \subitem Expanded applications include oral hygiene, food preservation, fish farming, and carcinogen degradation (Li et al. 2019).
\end{itemize}

\section{Xanthine Oxidoreductase}
\begin{itemize}
    \item Xanthine Oxidoreductase (XOR):
    \subitem Identified in bovine milk as an aldehyde reductase (Schardinger, 1902).
    \subitem Catalyzes oxidation of hypoxanthine to xanthine and xanthine to uric acid (Booth, 1938).

    \item Structural and Functional Aspects:
    \subitem XO (EC 1.17.3.2) and XD (EC 1.17.1.4) are interchangeable enzyme forms (Stirpe \& Della Corte, 1968, 1972).
    \subitem Widely distributed molybdenum-containing enzyme involved in purine degradation.

    \item XOR in Milk:
    \subitem Found in skim milk and cream (Kitchen, 1974), released from MFGM during phase inversion.
    \subitem Classified as a peripherally associated membrane protein (Briley \& Eisenthal, 1975).

    \item Presence in Milk Fractions:
    \subitem Detected in EVs and MFGs in both human and bovine milk.
    \subitem Proposed as a microRNA marker in commercial milk (Benmoussa et al. 2017).
\end{itemize}

\subsection*{Structure of Xanthine Oxidoreductase}
\begin{itemize}
    \item Structure and Redox Centers:
    \subitem XOR is a dimer of identical monomers, each with 1332 residues and four redox centers: 2Fe/2S clusters, FAD, and molybdopterin (Berglund et al. 1996).
    \subitem XO catalyzes xanthine oxidation with O$_2$ reduction, while XD uses NAD$^+$ (Hille, 2013).

    \item Interconversion Between XO and XD:
    \subitem XD is the native form; XO results from oxidation or proteolysis (Stirpe \& Della Corte, 1968).
    \subitem Conversion involves disulfide bond formation between Cys535-Cys992, altering NAD$^+$ binding (Nishino et al. 2005).
    \subitem XO/XD equilibrium is dynamic, with structural flexibility allowing interconversion (Nishino et al. 2015).

    \item XOR in Milk:
    \subitem Early assays suggested XOR absence in human milk, later disproven (Bradley \& Günther, 1960).
    \subitem Low XOR activity in human milk due to molybdenum deficiency (Godber et al. 1997).
    \subitem XOR activity in goat and sheep milk increases with dietary molybdenum supplementation (Atmani et al. 2004).

    \item Isolation and Purification:
    \subitem Early methods used pancreatin, causing irreversible XO conversion (Battelli et al. 1972).
    \subitem Alternative isolation methods include butanol partitioning and ion-exchange chromatography (Rajagopalan, 1986; Kristensen et al. 1996).
    \subitem Affinity chromatography with XO inhibitors improves purification (Beyaztaş \& Arslan, 2015).
\end{itemize}


\subsection*{Biological Role of Xanthine Oxidoreductase and Significance in Milk}
\begin{itemize}
    \item Biological Role of XOR:
    \subitem Housekeeping enzyme in purine catabolism; mutations cause xanthinuria (Bray, 1975; Ichida et al. 1997).
    \subitem Involved in innate immunity, linked to NF-$\kappa$B signaling (Vorbach et al. 2006).
    \subitem Generates ROS/RNS with LPO, contributing to bactericidal properties (Björck \& Claesson, 1979).

    \item XOR in Milk:
    \subitem Plays a structural role in milk fat globule secretion via butyrophilin interaction (Vorbach et al. 2002).
    \subitem Found in a stable complex with butyrophilin and adipophilin (Heid et al. 1996).
    \subitem Affects oral microbiome by releasing antibacterial compounds in neonatal saliva (Al-Shehri et al. 2015).

    \item Heat Stability and Industrial Impact:
    \subitem UHT treatment inactivates XOR (Ozturk et al. 2019).
    \subitem Not a reliable heat treatment marker, though debated (Griffiths, 1986; Andrews et al. 1987).
    \subitem Contributes to oxidative rancidity and aldehyde formation in milk (Aurand et al. 1967).

    \item XOR in Cheese Production:
    \subitem Inhibits *Clostridium tyrobutyricum* spores via ROS/RNS production (Vissers et al. 2007).
    \subitem Nitrate addition for spore control is controversial due to nitrosamine risks (Beresford et al. 2001).

    \item XOR Assays:
    \subitem Commonly measured via xanthine-to-urate conversion at 295 nm (Cerbulis \& Farrell, 1977).
    \subitem Alternative methods include HPLC and fluorometric detection of H$_2$O$_2$ (Rashidinejad et al. 2016; Hwang et al. 2016).
\end{itemize}

\section{$\gamma$-Glutamyltransferase}
\begin{itemize}
    \item $\gamma$-Glutamyltransferase (GGT) (E.C. 2.3.2.2):
    \subitem Catalyzes transfer of $\gamma$-glutamyl groups from glutathione to amino acids, peptides, or water.
    \subitem Found in cell membranes and suggested to aid amino acid supply for milk protein synthesis (Calamari et al. 2015).

    \item Presence in Milk:
    \subitem Primarily located in skim milk rather than MFGM (5.8:1 ratio, Kitchen 1974; 8:1 ratio, Reinhardt et al. 2013).
    \subitem 74.3\% of GGT activity detected in skim milk, only 6.6\% in MFGM (Baumrucker 1979).

    \item Detection Methods:
    \subitem Traditionally measured by p-nitroanilide release at 410 nm.
    \subitem Other methods include HPLC, fluorescence, and chemiluminescent probes for real-time detection (Ziobro \& McElroy 2013; An et al. 2019).
\end{itemize}


\subsection*{Structure of $\gamma$-Glutamyltransferase}
\begin{itemize}
    \item Structure of Bovine GGT:
    \subitem Sequence deduced from whole genome (UniProt G3N2D8).
    \subitem Expressed as a 568-residue precursor, autocleaved into a large (41.0 kDa) and small (19.8 kDa) subunit.
    \subitem Anchored to the plasma membrane via a single-pass transmembrane domain (residues 5–26).

    \item Catalytic Mechanism:
    \subitem Active site threonine (Thr380) facilitates autocleavage and acts as a catalytic nucleophile.
    \subitem Hydrolyzes $\gamma$-glutamyl amide bonds, releasing glutamate.
    \subitem Can perform transpeptidation, transferring $\gamma$-glutamyl groups to peptide acceptors (West et al. 2013).

    \item Purification and Structural Insights:
    \subitem Few studies on purification from milk; extensive work on rat kidney GGT (Castonguay et al. 2007).
    \subitem Human GGT crystal structure resolved at 1.67 Å (West et al. 2013).
\end{itemize}

\subsection*{Biological Role and Significance of Glutamyltransferase in Milk}
\begin{itemize}
    \item Biological Role of GGT:
    \subitem Hydrolyzes extracellular $\gamma$ -glutamyl compounds, maintaining intracellular glutathione levels.
    \subitem Involved in cysteine homeostasis, redox balance, and inflammation.
    \subitem Used as a diagnostic marker for liver disease, alcohol consumption, and metabolic disorders (Ndrepepa et al. 2018).

    \item Purification and Structural Properties:
    \subitem Purified from MFGM and skim milk membranes via a seven-step process (Baumrucker 1980).
    \subitem Consists of two glycosylated subunits (57 kDa and 25 kDa) with a total molecular mass of 80 kDa.
    \subitem Optimal activity at pH 8.5–9.0 and 45°C; inhibited by diisopropylfluorophosphate, iodoacetamide, Cu$^{2+}$, and Fe$^{3+}$ (Farkye 2003).

    \item Heat Stability and Industrial Relevance:
    \subitem More heat-resistant than alkaline phosphatase, but less than LPO.
    \subitem 50\% inactivation at 69°C after 3 min; residual activity found in raw milk cheese (Blel et al. 2002).
    \subitem Suggested as a marker for milk pasteurization (Vetsika et al. 2014).
\end{itemize}

\section{5'-Nucleotidase}
\begin{itemize}
    \item 5’-Nucleotidase (EC 3.1.3.5):
    \subitem Glycoprotein enzyme involved in nucleotide catabolism, hydrolyzing 5’-mononucleotides to nucleosides and phosphate (Reis 1934).
    \subitem Found in various tissues, bacteria, and plants with substrate specificity and pH variability (Zimmermann 1992).

    \item Structural and Functional Diversity:
    \subitem Exists in membrane-bound (ecto-5’-nucleotidase) and cytosolic forms, including mitochondrial variants (Misumi et al. 1990).
    \subitem Mammalian forms require Mg$^{2+}$ or Mn$^{2+}$, while bacterial forms are activated by Co$^{2+}$ and Ca$^{2+}$ (Neu 1967).

    \item Detection and Assays:
    \subitem Activity measured spectrophotometrically or via radioactive labeling (Edelson \& Duncan 1981).
    \subitem Commercial kits available using colorimetric detection of ammonia production.
    \subitem Specific inhibitors aid in distinguishing nucleotidase variants in lysates (Bianchi \& Spychala 2003).
\end{itemize}

\subsection*{Structure of 5'-Nucleotidase}
\begin{itemize}
    \item Structural Variants of 5’-Nucleotidase:
    \subitem Exists in multiple forms with distinct substrate affinities across organisms.
    \subitem Ecto-5’-nucleotidase forms homodimers, while cytosolic variants exist as monomers, dimers, or tetramers (Bailyes et al. 1984; Bianchi \& Spychala 2003).

    \item Structural Insights:
    \subitem Crystal structures determined for eukaryotic and bacterial 5’-nucleotidases (Walldén et al. 2007; Knöfel \& Sträter 1999).
    \subitem Human ecto-5’-nucleotidase structure resolved at 1.55–2.00 Å in open/closed conformations.
    \subitem Bovine ecto-5’-nucleotidase has 574 residues, 62,966 Da mass, and four disulfide bonds (Fini et al. 2000).

    \item Functional Features:
    \subitem Glycosylation patterns vary by species and tissue; contains four or five N-linked sites (Misumi et al. 1990).
    \subitem Active site located at the interface of N- and C-terminal domains, inaccessible to solvents.
    \subitem C-terminal contains substrate binding site and GPI anchor attachment (Knapp et al. 2012).
\end{itemize}

\subsection*{Biological Role and Significance of 5'-Nucleotidase in Milk}
\begin{itemize}
    \item 5’-Nucleotidase as a Marker:
    \subitem Used to study plasma membrane and milk secretion mechanisms.
    \subitem Inhibition of milk flow after concanavalin A injection suggests a role in exocytosis (Snow et al. 1980).

    \item Isolation and Properties:
    \subitem First isolated from bovine MFGM using detergents, precipitation, heat treatment, and chromatography (Huang \& Keenan, 1972).
    \subitem Exists in two isoforms with differing substrate affinities (K$_M$ values: 0.94 and 5 mM).
    \subitem Maximum activity at 69\textdegree C; loses 50\% activity after 60 min at 60\textdegree C.
    \subitem Extractable using n-butanol, with higher efficiency at lower temperatures (Ahn \& Snow, 1993).

    \item Detergent Sensitivity:
    \subitem Mild detergents (Triton X-100, Tween 20) enhance activity, while SDS inhibits it (Bhavadasan \& Ganguli, 1978).
    \subitem Higher concentrations of nonionic detergents inhibit bovine MFGM 5’-nucleotidase (Kanno \& Yamauchi, 1979).

    \item Potential Role in Milk:
    \subitem May aid nucleotide digestion in neonates by generating absorbable nucleosides (Janas \& Picciano, 1982).
    \subitem Nucleotide content is higher in human milk than bovine milk.
\end{itemize}

\section{Conclusion}
\begin{itemize}
    \item True Membrane-Associated Enzymes:
    \subitem Four enzymes are strongly linked to milk phospholipid structures: sulfhydryl oxidase, xanthine oxidoreductase, $\gamma$-glutamyltransferase, and 5’-nucleotidase.

    \item Enzymes Not Considered True Membrane Components:
    \subitem Catalase likely originates from somatic milk cells or microbial contamination.
    \subitem LPO lacks membrane-associated structural features; its presence in MFGM and EVs is likely due to fraction contamination.

    \item Other Enzymes:
    \subitem Alkaline phosphatase and lipoprotein lipase are membrane-bound but discussed elsewhere in the book.
\end{itemize}


