\chapter{Course Description}
\section{Content}
This course has the overall aim of providing students with a detailed knowledge of both theoretical and practical aspects of the molecular, colloidal and functional properties of milk and milk production. In addition, the functionality of food ingredients derived from milk will be covered. The course primarily deals with the following subjects:

Introduction: The chemistry of milk components such as lactose, milk proteins and peptides and lipids.

Milk production and its effects on milk: Milk secretion, the influence of housing and feeding regimes, milking technologies and their influence on raw milk quality, organic production of milk.

The colloidal system of milk: The properties of the casein micelles and the interplay with milk serum (whey) proteins and lipids.

The effects of heat treatment of milk: Denaturation and aggregation reactions, Maillard reactions, milk proteolytic enzymes and analytical separation processes, such as chromatography, used to characterise these. The effects of processes involving mainly heat treatment on the bioactivity of milk ingredients.

Functionality of milk constituents and methods applied for characterization of functionality: Solubility, gelation properties, rheology, microscopy, design of model systems and tests for functional properties. Biological activity of milk components.

\section{Learning Outcome}
After completing the course the student should be able to:

\subsection{Knowledge}
\begin{highlight}
    \begin{itemize}
        \item Describe in detail the composition of milk
        \item Describe the colloidal interactions governing stabilization and destabilization of milk and dairy products
        \item State the major factors influencing milk production and raw milk quality, including milk storage
        \item Outline the factors affecting milk ingredient functionality in food systems and how they can be analysed
        \item Appreciate the effects of dairy production and processing on milk at a molecular level.
    \end{itemize}
\end{highlight}

\subsection{Skills}
\begin{highlight}
    \begin{itemize}
        \item Apply the principles of colloid science to milk processing and milk ingredient functionality
        \item Structure a scientific presentation and produce a condensed synopsis.
\end{itemize}
\end{highlight} 

\subsection{Competences}  
\begin{highlight}
    \begin{itemize}
        \item Plan experiments related to the effect of a culinary technique on the sensory properties of food
        \item Discuss the factors influencing milk ingredient functionality
        \item Reflect on the health functionality/bioactivity of milk constituents
        \item Recognise the importance of working effectively in a group during laboratory practicals.
    \end{itemize}
\end{highlight}

\section{Litterature}
See Absalon for a list of course literature.

\section{Recommended Academic Qualifications}
Knowledge equivalent to a dairy internship is recommended.

Academic qualifications equivalent to a BSc degree is recommended.

\section{Teaching and Learning Methods}
The course consists of lectures and tutorials (4-8 hrs/week) and laboratory practicals (4-8 hrs/week), as well as e-learning, and excursions.

Lectures and tutorials provide an overview of milk quality and thermal processing on the functionality and bioactivity of milk and milk ingredients. Laboratory practicals dealing with aspects of milk as a colloidal system and milk protein functionality are also included as well as excursions.

Minor costs in relation to excursions are payed by the students.

\section{Remarks}
It is recommended to follow the course on the first year of the MSc Programme in Food Innovation and Health.

\section{Workload}
See table \ref{tab:workload} for an overview of the workload for the course.
\begin{table}
    \centering
    \caption{A table with an overview over the workload for the course.}
    \label{tab:workload}
    \rowcolors{2}{white}{gray!7}
    \begin{tabular}{ l | c}
        \textbf{Category} & \textbf{Hours} \\ 
        \hline
        Lectures & 26 \\ 

        Preparation & 90 \\

        Theory exercises & 15 \\ 

        Practical exercises & 32 \\
        
        E-Learning & 3 \\

        Excursions & 30 \\

        Exam & 10 \\ 
        \hline
        Total & 206 \\ 
    \end{tabular}
\end{table}

\section{Feedback Form}
Continuous feedback during the course of the semester
Peer feedback (Students give each other feedback)

\section{Sign Up}
Self Service at KUnet

http://www.science.ku.dk/english/courses-and-programmes/

https://www.science.ku.dk/english/continuing-and-professional-education/single-subject-courses/practical/

\section{Exam}
\begin{table}[h]
    \centering
    \caption{The table shows the details of the course exam, as defined from the website of the University of Copenhagen.}
    \label{tab:course_details}
    \rowcolors{2}{white}{gray!7}
    \begin{tabular}{ l | >{\raggedright\arraybackslash}p{\textwidth - 5.8cm} }
        \textbf{Category} & \textbf{Details} \\ 
        \hline
        Credit & 7.5 ECTS \\ 

        Type of assessment & Oral examination, 20 minutes \\ 

        Type of assessment details & One week prior to the actual exam, the student draws a subject and prepares an oral presentation in this subject. Assessment will be based on the oral presentation and following questions within the course curriculum. The presentation should take no more than 8 minutes. Following the presentation the student will be asked questions for approx. 7 minutes. \\ 

        Exam registration requirements & Approval of reports from practicals. \\ 

        Aid & All aids allowed \\ 

        Marking scale & 7-point grading scale \\ 

        Censorship form & No external censorship
        Several internal examiners \\ 

        Re-exam & Same as ordinary exam.

        The student draws a new subject a week before the re-exam and a new presentation is prepared and given at the examination. If the student has non-approved reports from the practicals, these must be handed in for approval at least two weeks prior to re-examination. \\ 
    \end{tabular}
\end{table}

\textbf{Criteria for exam assessment}
See Learning Outcome.
