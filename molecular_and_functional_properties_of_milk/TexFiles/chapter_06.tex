\chapter{Group Project - Enzymes - Membrane Associated Enzymes}
\setlength{\headheight}{12.71342pt}
\addtolength{\topmargin}{-0.71342pt}

\subsection*{Formal requirements for the summary}

\begin{itemize}
    \item The summary should not be longer than 3-4 pages excluding references
    \item The summary should be written as continuous text (a small number of bullet points may be used)
    \item All references must be cited with full bibliographic information. Citation style can be chosen individually, but should be consistent throughout the summary
    \item Please address the following aspects:
    \subitem An appropriate title and suitable headings
    \subitem Classification and modus operandi of the enzyme(s)
    \subitem The role of the enzyme(s) in milk and/or dairy products
    \subitem A short summary of 1-2 recent scientific studies related to the enzyme(s)
    \subitem References
    \item The deadline for submission is February 28
\end{itemize}

\section{Introduction}
\begin{itemize}
    \item Phospholipids (PL) in Milk: Estimates of PL content in bovine milk vary (0.9–2.3\% of total lipids) depending on extraction methods (Röse Gottlieb vs. Folch extraction).
    \item Milk Fat Globule Membrane (MFGM): Contains 60–65\% of milk PL, with the remaining 35–40\% found in the skim milk phase.
    \item Extracellular Vesicles (EVs): Recently recognized nano-sized phospholipid structures in skim milk, distinct from milk fat globules, but involved in intercellular communication.
    \item Membrane-Associated Enzymes: Identified in both MFGs and EVs, but their abundance is generally low. Many originate from ER, Golgi, or cytosolic crescents.
    \item Activity Considerations: Many of these enzymes remain inactive in milk due to the absence of substrates or unsuitable environmental conditions.
    \item Scope of Discussion: The summary will focus on enzymes relevant to mammary gland biology, milk integrity, and physiological functions upon consumption, excluding those related to lipid synthesis.
\end{itemize}

\section{Sulfhydryl Oxidase}
\begin{itemize}
    \item Sulfhydryl Oxidase (EC 1.8.3.2): Catalyzes oxidation of protein thiols (cysteine residues) to form disulfide bonds, reducing oxygen to hydrogen peroxide.
    \item Types in Milk: Exists in metal-dependent and flavin-dependent forms.
    \item Early Studies:
    \subitem Iron-dependent sulfhydryl oxidase (89-kDa, contains iron) was initially reported (Janolino \& Swaisgood, 1975).
    \subitem Later studies failed to confirm its presence (Jaje et al., 2007).
    \item Current Understanding:
    \subitem Flavin-dependent sulfhydryl oxidase (QSOX1) is well-documented in bovine milk.
    \subitem Sequence analysis confirmed it as part of the Quiescin-sulfhydryl oxidase family.
    \item Membrane Association:
    \subitem Initially believed to be strongly associated with phospholipid membranes (Kitchen, 1974).
    \subitem Later studies suggest a looser association, making it a more soluble protein (Jaje et al., 2007).
    \item Proteomic Evidence: QSOX1 has been identified in membrane fractions of both human and bovine milk (Liao et al., 2011; Reinhardt et al., 2013).
\end{itemize}

\subsection*{Structure of Flavin-Dependent Sulfhydryl Oxidase}
\begin{itemize}
    \item QSOX1 Splice Variants:
    \subitem QSOX1-L (Long form, 79.6 kDa): Contains a transmembrane region.
    \subitem QSOX1-S (Short form, 63.8 kDa): Lacks most of exon 12 and is more soluble.
    \subitem QSOX1-S is more prevalent than QSOX1-L, including in mammary-derived cell lines.

    \item Structural Features:
    \subitem Multi-domain enzyme derived from fusion of two ancient genes.
    \subitem Contains thioredoxin (Trx) domains, FAD-binding module, CxxC motifs (common in redox reactions).
    \subitem QSOX1-L has a membrane-spanning region, while QSOX1-S does not.

    \item QSOX1 in Bovine Milk:
    \subitem Jaje et al. (2007) isolated sulfhydryl oxidase from bovine skim milk, identifying it as QSOX1-L, though later studies suggest it was likely QSOX1-S.
    \subitem The enzyme migrated as a 62 kDa band in SDS-PAGE.

    \item Biological Significance:
    \subitem QSOX1 expression is linked to tumorigenesis (Antwi et al. 2009; Katchman et al. 2013).
    \subitem QSOX1-S has been isolated from mammalian blood serum (Israel et al. 2014).
    \subitem QSOX1-L can be proteolytically modified and secreted into the extracellular matrix, possibly after removal of its transmembrane domain (Rudolf et al. 2013).
    \subitem Dimerization of QSOX1 has been demonstrated in the same study.
\end{itemize}

\subsection*{Biological Role of Flavin-Dependent Sulfhydryl Oxidase and Significance in Milk}

