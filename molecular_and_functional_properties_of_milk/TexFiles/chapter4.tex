\chapter{Lecture Exercises}
\setlength{\headheight}{22.94003pt}
\addtolength{\topmargin}{-10.22661pt}

\section{19.11.24 - Exercise 1 - Carrot Soup}
You are working in a company that produces frozen soups, ready to heat and eat. You are in the process of developing a new product, a carrot soup with an orange flavour. The basic recipe looks like this:

\begin{table}
    \centering
    \caption{Basic Recipe for Carrot Soup}
    \label{tab:carrot_soup}
    \begin{tabular}{l|p{10cm}}
    \textbf{Ingredient} & \textbf{Preparation} \\
    \hline
    Oil & • Heat a large saucepan over medium heat. Add oil. Cook the  \\
    \cline{1-1}
    1 Onion, finely choped & onion, stirring occasionally, for 3 minutes or until soft. Add \\
    \cline{1-1}
    1 kg carrots, finely chopped & carrot and cook, stirring often, for 5 minutes or until just soft. \\
    \cline{1-1}
    1L chicken stock & • Add stock. Bring to boil. Peel 2 strips of rind from the orange.  \\
    \cline{1-1}
    1 Orange & Add to pan. Reduce heat to medium-low. Simmer for 20 min. \\
    \cline{1-1}
    1 tbsp parsley & or until carrot is tender. Remove and discard orange rind. \\
    & • Set aside for 5 minutes to cool. Use a blender to puree until \\
    & smooth. Juice orange. Add to soup. Place over medium heat \\
    & and cook for 2 minutes or until heated through. Top with parsley. \\

    \end{tabular}
\end{table}

\subsection*{a.}
Describe the structure of the cell wall of the carrot.
\subsection*{b.}
Name the three major polysaccharides in the cell wall of the carrot and describe what happens with these polysaccharides during cooking.

\subsection*{c.}
The orange you have used for the soup is fairly ripe, what kind of pectic substances do you think predominate in the orange? Explain also briefly the changes that the pectic substances undergo during ripening. Explain also whether you think the pectic substances extracted from the orange will contribute to gel formation in the soup, and why /why not.

\subsection*{d.}
Name the pigment that gives a nice orange colour to the soup and explain where it is found in the plant cell and describe its solubility properties.

\subsection*{e.}
You consider improving flavour by adding lemon juice to the soup as well as orange juice. To make it a little simpler you want to add all ingredients at the onset of cooking. How do you think that addition of citrus juice might affect the softening of the carrots, if the juice is added at the onset of cooking? Explain why.




